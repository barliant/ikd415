\documentclass[a4paper,12pt,answers]{exam}
%% \documentclass[a4paper,12pt]{exam}
\usepackage[bahasa]{babel}
\usepackage{graphicx}
\usepackage{fancyvrb}
\usepackage{epic}
\usepackage{ecltree}
%\usepackage{pstricks,pst-node}
%\usepackage{vaucanson-g}
%\usepackage[a4paper]{geometry}
\usepackage{wordlike}
\usepackage{xifthen}
\usepackage{algorithmic}

%%%%%%% Packages initializations
%% Paper sizes
%\geometry{left=3cm}
%\geometry{top=3cm}
%\geometry{right=3cm}
%\geometry{bottom=3cm}

\fvset{fontsize=\scriptsize,numbers=left,frame=leftline}
%\fvset{fontsize=\scriptsize,frame=leftline}

%% Season-based settings and Conditionals, xifthen
\newboolean{SemesterReguler}
\newboolean{Quiz}
\newboolean{AdaPejabatJurusan}
\newboolean{JawabanLangsung}
\newboolean{NamaAlias}
\newboolean{AdaKodeSoal}

\setboolean{SemesterReguler}{true}
\setboolean{Quiz}{true}
\setboolean{AdaPejabatJurusan}{false}
% Jawaban ditulis di lembar soal? true kalo iya. Jangan lupa menyalakan opsi answer.
\setboolean{JawabanLangsung}{true}
\setboolean{NamaAlias}{false}
\setboolean{AdaKodeSoal}{false}

% NamaAlias muncul jika:
% Quiz, atau
% Ujian, dan JawabanLangsung

%% Locale
\newcommand{\universitas}{Universitas Trisakti}
\newcommand{\fakultas}{Fakultas Teknologi Industri}
\newcommand{\programstudi}{Teknik Informatika, Sistem Informasi}
\newcommand{\namakuliah}{Intelijen Data}
\newcommand{\kodekuliah}{IKD415}
\newcommand{\tipeujian}{Kuis 1}
\newcommand{\semester}{Semester Gasal 2016/2017}
\newcommand{\tanggalujian}{Jumat, 30 September 2016}
\newcommand{\waktuujian}{30 menit}
\newcommand{\sifatujian}{Buka buku}
\newcommand{\dosenkoordinator}{(Anung B Ariwibowo, MKom)}
\newcommand{\kodesoal}{\fbox{Soal B}}

\ifthenelse{\boolean{AdaPejabatJurusan}}{
	\newcommand{\kajursekjur}{(Ahmad Zuhdi, SSi, MKom)}
}
{
	\newcommand{\kajursekjur}{(\ldots\ldots\ldots\ldots\ldots\ldots\ldots\ldots\ldots)}
}
\newcommand{\ralat}[1]{\textbf{(RALAT: #1)}}

\renewcommand{\solutiontitle}{\noindent\textbf{Jawab:}\par\noindent}
\renewcommand{\partlabel}{\thepartno.}
\vqword{Nomor}
\hqword{Nomor}
\vpword{Poin}
\hpword{Poin}
\vsword{Nilai}
\hsword{Nilai}
\pointsinmargin
\boxedpoints
\addpoints

\pagestyle{headandfoot}
\runningheadrule
\footrule
%\header{\namakuliah \\ \kodekuliah}{}{60.Kul.1TIF/R.0 \\ \tanggalujian}
\ifthenelse{\boolean{AdaKodeSoal}}{
  \firstpageheader{\kodesoal}{}{60.Kul.1TIF/R.0}
}
{
  \firstpageheader{}{}{60.Kul.1TIF/R.0}
}
\runningheader{\namakuliah \\ \kodekuliah}{}{60.Kul.1TIF/R.0 \\ \tanggalujian}
%%\lhead[]{\namakuliah \\ \kodekuliah}
%%\rhead[]{60.Kul.1TIF/R.0 \\ \tanggalujian}
\cfoot{\iflastpage{Halaman \thepage\ dari \numpages\\Kelulusan bukan ditentukan oleh hasil kerja teman anda.}{Halaman \thepage\ dari \numpages}}
% Selamat bekerja dengan kejujuran

%%\cfoot{\iflastpage{Halaman \thepage\ dari \numpages\\Orang yang dicontek tidak lebih tahu daripada Orang yang mencontek.}{Halaman \thepage\ dari \numpages}}

%% package ecltree
\setlength{\GapWidth}{10mm}
%%%%%%% Packages initializations

\begin{document}
\ifthenelse{\boolean{Quiz}}{
	% SEMESTER PENDEK or QUIZ
	\begin{center}
	% \begin{coverpages}
		{\bf
			\parbox{5.5in}{\centering
				{\Large
					\fakultas\ -- \universitas
					
					\programstudi
				}
				\vspace{0.5cm}
				
				{\large
					\semester
					
					\namakuliah\ -- \kodekuliah
				}
				\vspace{0.5cm}
				
				{\small
					\tipeujian , \tanggalujian\
					
					\waktuujian\ -- \sifatujian
				}
				\vspace{1.5cm}
			}
		}
		\vspace{0.5cm}
		\hbox to \textwidth{Nama : \enspace\hrulefill}
		\vspace{0.5cm}
		\hbox to \textwidth{NIM : \enspace\hrulefill}
		\ifthenelse{\boolean{NamaAlias}}{
			\vspace{0.5cm}
			\hbox to \textwidth{e-mail: \enspace\hrulefill}
			\vspace{0.5cm}
			\hbox to \textwidth{Nomor HP aktif: \enspace\hrulefill}
			\vspace{0.5cm}
			\hbox to \textwidth{Nama alias \footnote{Nama alias digunakan untuk mengumumkan nilai.}: \enspace\hrulefill}
		}
		
		\vspace{1.0cm}
		\gradetable[h][questions]
	% \end{coverpages}
	\end{center}
}
{
	% EXAM
	\begin{center}
	  {\bf
	    {\centering {\Large
	      \fakultas\ -- \universitas } \\ \large{ \programstudi } \\ \normalsize{\tipeujian \\ \semester }
	    }
	    \vspace{0.5cm}
	    \hrule
	    {\small
	      \vspace{1cm}
	      \begin{tabular}{p{0.5\textwidth}p{0.3\textwidth}}
	        \namakuliah\ -- \kodekuliah & \multicolumn{1}{r}{\sifatujian} \\
	        \tanggalujian\ & \multicolumn{1}{r}{\waktuujian} \\
	        & \\
	        \multicolumn{1}{c}{\emph{Koordinator Mata Kuliah}} &
	        \multicolumn{1}{c}{\emph{Diperiksa dan Disetujui oleh}} \\
	        & \multicolumn{1}{c}{Ketua Progam Studi / Sekjur} \\
	        & \\
	        & \\
	        & \\
	        \multicolumn{1}{c}{\dosenkoordinator} & \multicolumn{1}{c}{\kajursekjur} \\
	      \end{tabular}
	    }
	  }
	  \vspace{0.5cm}
	  \hrule
	  \vspace{0.1cm}
	  Ujian ini terdiri atas \numpages~halaman, \numquestions~soal, dan \numpoints~poin.
	  \ifthenelse{\boolean{JawabanLangsung}}{
			\vspace{0.5cm} %\hbox to \textwidth{Nama : \enspace\hrulefill}
			Nama : \enspace\hrulefill
			\vspace{0.5cm} %\hbox to \textwidth{NIM : \enspace\hrulefill}
			NIM : \enspace\hrulefill
			
			\vspace{0.2cm} {\footnotesize \gradetable[h][questions]}
		}
	\end{center}
}

% A 1 2 3 4
% B 2 1 3 4
% C 2 1 4 3
% D 3 1 2 4
\begin{questions}
  \ifthenelse{\boolean{NamaAlias}}{
  \question Tuliskan nama alias, jika anda menghendaki identitas anda disamarkan saat nilai sementara diumumkan sepanjang perkuliahan berlangsung.
  }
  
  
  \question[10] Diberikan dua buah titik dalam ruang \emph{Cartesian} $A = (10, 8)$ dan $B = (-9, 4)$. Tuliskan persamaan garis yang melalui kedua titik tersebut.
  \begin{solution}
    \vspace{5.0cm}
  \end{solution}
  
  
  \question[10] Dari sekian banyak algoritme yang ada, terdapat dua kelompok algoritme \emph{Machine Learning}, yakni \emph{Penalized Regression} dan \emph{Ensemble Methods}. Jelaskan perbedaan di antara kedua jenis algoritme tersebut dari sudut pandang \textbf{akurasi} dan \textbf{waktu pembelajaran}.
  \begin{solution}
    \vspace{5.0cm}
  \end{solution}
  
  
  \question[10] Diketahui sebuah \emph{neural network} yang tersusun atas \emph{perceptron} memiliki dua lapis, \emph{input layer} dan \emph{output layer}. Perceptron di input layer terdiri dari 4 neuron, dan di output layer terdapat 2 neuron. Jika diketahui nilai \emph{bias} adalah $b = 2$ matriks bobot di antara input layer dan output layer adalah
  \[
  \left(\begin{array}{cc}
  -2 & 2 \\
  2 & -2 \\
  1 & 2 \\
  2 & 1 \\
  \end{array}
  \right)
  \]
  Tuliskan sinyal yang dikeluarkan oleh kedua neuron di output layer, jika neural network tersebut mendapatkan sinyal input $(1, 0, 1, 0)$.
  \begin{solution}
    \vspace{5.0cm}
  \end{solution}
\end{questions}
\end{document}
